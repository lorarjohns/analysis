\documentclass[12pt]{scrartcl} % Font size

%%%%%%%%%%%%%%%%%%%%%%%%%%%%%%%%%%%%%%%%%
% Original from:
% Frits Wenneker
% CC BY-NC-SA 3.0 (http://creativecommons.org/licenses/by-nc-sa/3.0/)
%%%%%%%%%%%%%%%%%%%%%%%%%%%%%%%%%%%%%%%%%

\usepackage{amsmath, amsfonts, amsthm} % Math packages

\usepackage{cancel}

\usepackage[shortlabels]{enumitem}

\usepackage{listings} % Code listings, with syntax highlighting

\usepackage[english]{babel} % English language hyphenation

\usepackage{graphicx} % Required for inserting images
\graphicspath{{Figures/}{./}} % Specifies where to look for included images (trailing slash required)

\usepackage{booktabs} % Required for better horizontal rules in tables

\numberwithin{equation}{section} % Number equations within sections (i.e. 1.1, 1.2, 2.1, 2.2 instead of 1, 2, 3, 4)
\numberwithin{figure}{section} % Number figures within sections (i.e. 1.1, 1.2, 2.1, 2.2 instead of 1, 2, 3, 4)
\numberwithin{table}{section} % Number tables within sections (i.e. 1.1, 1.2, 2.1, 2.2 instead of 1, 2, 3, 4)

\setlength\parindent{0pt} % Removes all indentation from paragraphs

\setlist{noitemsep} % No spacing between list items


%	DOCUMENT MARGINS

\usepackage{geometry} % Required for adjusting page dimensions and margins

\geometry{
	paper=a4paper, % Paper size, change to letterpaper for US letter size
	top=2.5cm, % Top margin
	bottom=3cm, % Bottom margin
	left=3cm, % Left margin
	right=3cm, % Right margin
	headheight=0.75cm, % Header height
	footskip=1.5cm, % Space from the bottom margin to the baseline of the footer
	headsep=0.75cm, % Space from the top margin to the baseline of the header
	%showframe, % Uncomment to show how the type block is set on the page
}

%	FONTS

\usepackage[utf8]{inputenc} % Required for inputting international characters
\usepackage[T1]{fontenc} % Use 8-bit encoding

\usepackage{fourier} % Use the Adobe Utopia font for the document

%	SECTION TITLES

\usepackage{sectsty} % Allows customising section commands

\sectionfont{\vspace{6pt}\centering\normalfont\scshape} % \section{} styling
\subsectionfont{\normalfont\bfseries} % \subsection{} styling
\subsubsectionfont{\normalfont\itshape} % \subsubsection{} styling
\paragraphfont{\normalfont\scshape} % \paragraph{} styling

%	HEADERS AND FOOTERS

\usepackage{scrlayer-scrpage} % Required for customising headers and footers

\ohead*{} % Right header
\ihead*{} % Left header
\chead*{} % Centre header

\ofoot*{} % Right footer
\ifoot*{} % Left footer
\cfoot*{\pagemark} % Centre footer

% COMMANDS AND SYMBOLS

\newcommand{\enumalpha}{
	\begin{enumerate}[label=(\alph*)]
		
	} 

\title{	
	\normalfont\normalsize
	\textsc{Analysis with an Introduction to Proof}\\ % Your university, school and/or department name(s)
	\vspace{25pt} % Whitespace
	\rule{\linewidth}{0.5pt}\\ % Thin top horizontal rule
	\vspace{20pt} % Whitespace
	{\huge Sets and Functions, Chapter Exercises}\\ % The assignment title
	\vspace{12pt} % Whitespace
	\rule{\linewidth}{2pt}\\ % Thick bottom horizontal rule
	\vspace{12pt} % Whitespace
}

\author{\LARGE L. J.} % Your name

\date{\normalsize\today} % Today's date (\today) or a custom date

\newtheorem*{theorem}{Theorem}

\begin{document}

\maketitle % Print the title

\section{Section 1}

\nsubsection{2}{Mark each statement True or False. Justify each answer.}
\begin{enumerate}[(a)]
	\item \emph{False}. Let $A = \{x \in \mathbb{R}: x \text{ is prime}\}$ and let $B = \{x \in \mathbb{R}: x \text{ is divisible by 6} \}$. Then $A \cap B = \emptyset$, but $A \neq \emptyset$ and $B \neq \emptyset$.
	\item \emph{True}. Let $A = \{1, 3, 5\}$ and let $B = \{2, 4, 6\}$. Then $A \cup B = \{1, 2, 3, 4, 5, 6\}$. If $x \in A \cup B$, then $x$ is equal to one of those values. If $x = 4$, then it is also true that $x \in B$. By the same logic more generally, if $x \in A \cup B$, then $x \in A$ or $x \in B$.
	\item \emph{False}. If $x \in A\setminus B$, then $x \in A$ \emph{and} $x \notin B$. Let $C = \{x: (x \in A) \land (x \in B)\}$. If $x \in C$, then if $x \in A\setminus B \stackrel{?}{\implies} (x \in A) \lor (x \notin B)$, it is also true $x \in A\setminus B$. But this is at odds with the definition of the relative complement of $B$ with respect to $A$.
	\item \emph{False}. It's fine to begin this way; the only nontrivial proof is one in which $S$ is nonempty.
\end{enumerate}

\nsubsection{4}{Let $A = \{2,4,6,8\}$, $B =\{6,8,10\}$, and $C = \{5,6,7,8\}$. Find the following sets.}
\begin{enumerate}[(a)]
	\item $\{6,8\}$
	\item $\{2,4,6,8,10\}$
	\item $\{2,4\}$
	\item $\{6,8\}$
	\item $\{10\}$
	\item $\{5,7,10\}$
	\item $\emptyset$
	\item $\{5,7\}$
\end{enumerate}

\nsubsection{6}{Let A and B be subsets of a universal set U. Simplify each of the following expressions.}
\begin{enumerate}[(a)]
	\item $U$
	\item $\emptyset$
	\item $A \cap B$
	\item $A \cup B$
	\item $A$
	\item $A$
\end{enumerate}

\nsubsection{8}{Let $S = \{\emptyset, \{\emptyset\}\}$. Determine whether each of the following is True or False. Explain your answers.}
\begin{enumerate}[(a)]
	\item \emph{True}. $\forall x \in \emptyset$, $x \in S$. Therefore, $\emptyset \subseteq S$.
	\item \emph{True}. $S$ is strictly enumerated, and $\emptyset$ is one of its elements.
	\item \emph{True}. $\{\emptyset\} \subseteq S$ because for all $x \in \{\emptyset\}$, $x \in S$. Unlike in (a), though, the proof is not vacuous; $\emptyset$ is an element of both sets.
	\item \emph{True}. $\{\emptyset\}$ is an element in $S$, as defined by enumeration. (To contrast (c) and (d), $\{\{\emptyset\}\}$ would also be a valid subset of $S$.)
\end{enumerate}
\nsubsection{10}{Fill in the blanks in the proof of the following theorem.}
\begin{theorem}
$A \subseteq B$ iff $A \cup B = B$.
\end{theorem}
\begin{proof}
Suppose that $A \subseteq B = B$. If $x \in A \cup B$, then $x \in A$ or $x \in B$.
Since $A \subseteq B$, in either case, we have $x \in B$. Thus $A \subseteq B$.
On the other hand, if $x \in B$, then $x \in A \cup B$, so $A \subseteq B$. Hence $A \cup B \ B$. \\
Conversely, suppose that $A \cup B = B$. If $x \in A$, then $x \in A \cup B$. But $A \cup B = B$, so $x \in B$. Thus $A \subseteq B$.
\end{proof}
\nsubsection{14}{Which statements below would enable one to conclude that $x \in A \cup B$?}
\begin{enumerate}[(a)]
	\item If $x \in A$ and $x \in B$, then $x \in A \cup B$.
	\item If $x \in A$ or $x \in B$, then $x \in A \cup B$.
	\item This statement tells us $A \subseteq B$, but does not let us conclude anything about $x$.
	\item If $x \notin A$, then $x \in B$ implies that if $x \notin B$, then $x \in A$. Since $x$ must always be either in $A$ or $B$, we can conclude that $x \in A \cup B$.
\end{enumerate}

\nsubsection{16}{Which statements below would enable one to conclude that $x \in A\setminus B$?}
\begin{enumerate}[(a)]
	\item If $x \in A$ and $x \notin B\setminus A$, then $x$ could be in $A \cap B$, so we cannot conclude that $x \in A \setminus B$.
	\item If $x \in A \cup B$ and $x \notin B$, then $(A \cup B) \cap U\setminus B$. Equivalently, $A \cap U\setminus B \cup B \cap U\setminus B$, and $A\setminus B \cup \emptyset$, or $A\setminus B$. Therefore we can conclude that $x \in A\setminus B$.
	\item If $x \in A \cup B$ and $x \notin A \cap B$, then $x \in A$ or $x \in B$. If $x \in A$, then $x \in A\setminus B$. But if $x \in B$, then $x \notin A\setminus B$. So we cannot conclude that $x \in A\setminus B$.
	\item If $x \in A$ and $x \notin A \cap B$, then $x \notin B$. Thus $x \in A\setminus B$
\end{enumerate}

\nsubsection{18}{Prove that the empty set is unique. That is, suppose that $A$ and $B$ are empty and prove that $A = B$.}
\begin{proof}
Suppose that $A$ and $B$ are empty sets. If $A = B$, then $A \subseteq B$ and $B \subseteq A$. \\
If $A \subseteq B$, then if $x \in A$, $x \in B$. By the definition of an empty set, for all $x$, $x \notin \emptyset$. Therefore, for all $x$, $x \notin A$ and $x \notin B$. By logical implication, then, if $x \in A$, $x \in B$. This implies that $A \subseteq B$. \\
On the other hand. if $x \in B$, then $x \in A$. By the same argument, we conclude $B \subseteq A$. So, we have $A \subseteq B$ and $B \subseteq A$. Thus $A = B$.
\end{proof}

\nsubsection{20}{Prove: $A \cap B$ and $A\setminus B$ are disjoint and $A = (A \cap B) \cup (A \setminus B)$.}
\begin{proof}
Assume the conclusion that $(A \cap B) \cap (A\setminus B) = \emptyset$. \\
If $x \in (A \cap B)$, then $x \in A$ and $x \in B$. But then $x \notin (A \setminus B)$, because if $x \in A \setminus B$, then $x \in A$ and $x \notin B$. In other words, if $x \in (A \cap B)$, $x \in (A \setminus (A \setminus B))$. So $(A \cap B) \nsubseteq (A \setminus B)$. \\
Conversely, if $x \in (A\setminus B)$, then $x \in A$ and $x \in B$. But then $x \notin (A \cap B)$. So $(A\setminus B) \nsubseteq (A \cap B)$. Thus, $(A \cap B) \cap (A\setminus B) = \emptyset$. Therefore $(A \cap B)$ and  $(A\setminus B)$ are disjoint. \\
In other words:
\begin{align}
(A \cap B) \cap (A\setminus B) \\
= (A \cap B) \cap (A \cap U\setminus B) \\
= (U\setminus A \cup U\setminus B) \cup (U\setminus A \cup B) \\
= (U\setminus A) \cup (U\setminus B \cup B) \\
= A \cap (B \cap U\setminus B) \\
= A \cap \emptyset \\
= \emptyset
\end{align}
\end{proof}
\begin{proof}
To prove  $A = (A \cap B) \cup (A \setminus B)$, first show that if $x \in (A \cap B) \cup (A\setminus B)$, then $x \in A$. \\
If $x \in (A \cap B) \cup (A\setminus B)$, then $x \in (A \cup A\setminus B)$ and $x \in (B \cup A\setminus B)$. Suppose $x \in (A \cap B) \cup (A\setminus B)$.
If $x \in A$, then $x \in (A \cup A\setminus B)$ and $x \in (B \cup A\setminus B)$, so $x \in (A \cap B) \cup (A\setminus B)$. \\If $x \in B$, then $x \in (B \cup A\setminus B)$ (equivalently, $(A \cup B)$), but $x \notin (A \cup A\setminus B)$ (equivalently, $A$). So by necessity, we have $x \in A$, and $(A \cap B) \cup (A\setminus B) = A$. \\

On the other hand, suppose $x \in A$. Then $x \in (A \cap B)$ or $x \in (A\setminus B)$:
\begin{align}
A\\
= A \cap U \\
= A \cap (B \cup U\setminus B) \\
= (A \cap B) \cup (A \cap U\setminus B) \\
= (A \cap B) \cup (A\setminus B)
\end{align}
\\Either way, $x \in (A \cap B) \cup (A \setminus B)$. So we have $A = (A \cap B) \cup (A \setminus B)$.

\end{proof}

\section{Section 2}

\nsubsection{2}{Mark each statement True or False. Justify each answer.}
\begin{enumerate}[(a)]
	\item \emph{True}. If $\mathscr{P}$ is a partition of $S$, then if $x$ and $y$ are in the same part of the partition, we can define $x R y$ as an equivalence relation such that the equivalence classes are the same as $\mathscr{P}$.
	\item \emph{False}. If $xRx$ for all $x \in S$, then $R$ is reflexive. But not all relations are reflexive.
	\item \emph{False}. A partition of $S$ can be defined by $\{E_x: x \in S \}$, where $E_x$ represents an equivalence class. An equivalence class is defined by $E_x = \{y \in S: yRx\}$.
	\item \emph{True}. By definition, a partition $\mathscr{P}$ consists of nonempty, disjoint sets such that for all $x \in S$, $x$ belongs to some subset of $\mathscr{P}$.
\end{enumerate}

\nsubsection{4}{Show that $\{a\} \times \{a\} = \{\{\{a\}\}\}$.}
$(a,b)$  is the set whose members are $\{a\}$ and $\{a,b\}$.
	\begin{align*}
	(a,b) &= \{\{a\}, \{a,b\}\}
	\end{align*}
	And the product of two sets $A$ and $B$ is
	\begin{align*}
	A \times B \ \{(a,b) &: a \in A \text{and} b \in B \}
	\end{align*}		
	Then,
	\begin{align*}
	\{a\} \times \{a\} &= \{(a,a)\} \\
	&= \{\{ \{a\}, \{a, a\} \}\}.
	\end{align*}
	Since $a = a$, $\{a, b\} = \{a\}$, and the above simplifies to
	\begin{align*}
	&= \{\{\{a\} \{a\}\}\} \\
	&= \{\{\{a\}\}\}.
	\end{align*}

\nsubsection{6}{Let $A$ be be any set and let $B=\emptyset$. What can you conclude about $A \times B$?}
	Let $A$ be any set, and let $B$ be $\emptyset$. 
	By definition, 
	\begin{align*}
	A\times B = \{(a,b): a \in A \text{ and } b \in B \}
	\end{align*}
	
and
	\begin{align*}
	\emptyset = \{x: &\forall x, x\notin \emptyset\}.
	\end{align*}
	
	
	Then if $B = \emptyset$, the product $A\times B$ requires that all the ordered pairs in the product have a first element in $A$ and a second element in $\emptyset$. But now, $\forall b, b \notin B$, so there are no ordered pairs that satisfy the definition of the product of $A \times B$.  \\
	Therefore, $A \times B= \emptyset$.
	
	
\nsubsection{8}{Let $A = \{1, 2\}$.}
\begin{enumerate}[(a)]
\item \textbf{How many elements are in the set $A \times A$?}
$\mathopen|A \times A \mathclose| = 4$.
\item \textbf{How many possible relations are there on the set $A$?}
A relation on $A$ is any subset $R$ of $A \times A$ ($R \subset A \times A$); any element can be in or out of the relation. So, what is the power set of $A$?
$$ 2^{\mathopen|A \times A \mathclose|} = 2^4 = 16 $$
\item \textbf{How many possible relations are there on the set $\{1, 2, 3\}$?}
For $A$, we had $\mathopen|A \mathclose| = 2$. Let $S$ be a generic set, $n = \mathopen| S \mathclose|$ be the number of elements in the set, and $\mathopen| R \mathclose|$ be the total number of possible relations on $S$. Then for $S_n$, we have $\mathopen| R \mathclose| = 2^{n^2}$. If $S = \{1, 2, 3\}$, then $n = 3$ and we have
$$ \mathopen| R \mathclose| = 2^{(3^2)} = 2^9 = 512 $$ possible relations.
\end{enumerate}

\nsubsection{10}{Prove or give a counterexample.}
\begin{enumerate}[(a)]
\item Let $A = \{2, 3\}$ and $B = \{0, -5\}$.
Then $$ A \times B = \{(2,0), (2,-5), (3,0), (3,5)\} $$ and 
$$ B \times A = \{(0,2), (0,3), (-5,2), (-5,3)\}. $$ Therefore $$ A \times B \neq B \times A. $$
\item $(A \cup B) \times C = (A \times C) \cup (B \times C)$. \\
Let $$ A = \{x: x \in [0, 3)\} $$
$$ B = \{x: x \in (2, 4]\} $$
$$ C = \{x: x \in [-2, 2]\}. $$

Then $$ (A \cup B) \times C = \{(x,y): x \in [0, 4] \text{ and } y \in [-2, 2] $$

We have:
$$ A \times C = \{(x, y): x \in [0,3) \text{ and } y \in [-2,2]\} $$
$$ B \times C \ \{(x, y): x \in (2, 4] \text{ and } y \in [-2, 2]\} $$

and 

$$ (A \times C) \cup (B \times C) = \{(x, y): x \in [0, 4] \text{ and } y \in [-2, 2] \}.$$
\item TK
\item TK
\end{enumerate}
	
\nsubsection{12}{Find examples of relations with the following properties.}
Let $S = \{p, q, r\}$. Then define the following relations $R$ on $S$:
\begin{enumerate}[(a)]
\item \textbf{$R$ is reflexive, but not symmetric or transitive.} \\ $R = \{(p,p),(q,q),(r,r,)\}$
\item \textbf{$R$ is symmetric, but not reflexive or transitive.} \\ $R = \{(p,q), (q,p), (r,q), (q,r)\}$
\item \textbf{$R$ is transitive, but not symmetric or reflexive.} \\ $R = \{(p,q), (q,r), (p,r)\}$
\item \textbf{$R$ is reflexive and symmetric, but not transitive.} \\ $R = \{(p,p), (q,q), (r,r), (p,q), (q,p), (q,r), (r,q)\}$
\item \textbf{$R$ is reflexive and transitive, but not symmetric.} \\ $R = \{(p,p), (q,q), (r,r), (p,q), (q,r), (p,r)\}$
\item \textbf{$R$ is symmetric and transitive, but not reflexive.} \\ $R = \{(p, q), (q, p), (q, r), (r, q), (p, r), (r, p)\}$
\end{enumerate}

\nsubsection{14}{Let $S = \mathbb{R} \times \mathbb{R}$. Verify that the relation $(a,b)R(c,d)$ iff $a + d = b + c$ is an equivalence relation. Describe the equivalence class $E_{(7,3)}$.}

Rearrange the relation as:
\begin{align*}
a+d &= b+c \\
a+d-c &= b \\
a-c &= b-d
\end{align*}

The relation is reflexive: $(a,b)R(a,b)$ gives
$$ a-a = b-b $$
$$ 0 = 0 $$

The relation is symmetric:
\begin{align*}
(c,d) &R (a,b) \\
c-a &= d-b \\
-c+a &= -d+b  \\
a-c &= b-d
\end{align*}

The relation is transitive:
$$(a,b)R(c,d) \land (c,d)R(m,n) \implies (a,b)R(m,n)$$

\begin{align*}
TK TK TK TK
\end{align*}

The equivalence class $E_{(7,3)} = \{d \in S: d = c-4\}$:
\begin{align*}
7 + d &= 3 + c \\
d &= c - 4
\end{align*}
It describes a linear relationship.

\nsubsection{16}{Let $S = \mathbb{R} \times \mathbb{R}$. Define the equivalence relation $R$ on $S$ by $(a,b)R(c,d)$ iff $b-3a = d - 3c$.}

\begin{enumerate}[(a)]

\item We can rearrange the relation as:

\begin{align*}
b-d &= 3a - 3c \\
b-d &= 3(a-c) \\
\end{align*}

So, $R$ partitions $S$ into parallel lines of the form $y = 3x$ $(+ k)$.

\item $E_{(2,5)}$ would be the part of the partition containing the point $(2,5)$, which we obtain by substituting it for one-half of the ordered pairs:

\begin{align*}
b-3a &= 5 - 3(2) \\
b-3a &= -1
x &= 3a -1
\end{align*}

So the equivalence class is described by the linear relationship $y = 3x -1$.
\end{enumerate}

\nsubsection{18}{Let $R$ be the relation $\{(1,1), (2,2), (3,1), (3,3)\}$ on the set $\{1,2,3\}$. If $R$ is an equivalence relation, list the pieces of the partition determined by $R$, and if not, state why.}

$R$ is not an equivalence relation because it is not symmetric. $(3,1) \in R$ but $(1,3) \notin R$.

\nsubsection{20}{Let $S = \{a, b, c, d\}$ and let $P = \{\{a\}, \{b, c, d\}\}$. Describe the equivalence relation $R$ on $S$ determined by $P$ by listing the ordered pairs in the relation.}

We see that $a$ is in a partition alone, but $b, c$ and $d$ are in a relation with each other, so the relation consists of the pairs

\begin{align*}
\{ &(a,a), \\
&(b,b), (c,c), (d,d), \\
&(b,c), (b,d), \\
&(c,b), (c,d), \\
&(d,c), (d,b) \}.
\end{align*}

\section{Section 3}

%2.3 #2, 4, 6, 8, 10, 14, 16, 20 (8 problems)

\nsubsection{2}{Mark each statement True or False. Justify your answer.}
\begin{enumerate}[(a)]
\item \emph{True}. By the existence requirement of the definition of a function, for all $a$ in $A$, there must exist some $b$ in $B$ such that $f(a) = b$. Since $C \subset A$, it follows that every $c$ in $C$ is also in $A$, and so for all $c$ in $C$, there must be some $b$ in $B$ such that $f(c) = b$. Thus $f(C)$ is a nonempty subset of $B$.
\item \emph{False}. If $f$ is surjective, it is a mapping form all elements in the domain $A$ to all elements in $B$, that is to say the range of $f$ is equal to its codomain. But though $f^{-1}(y)$ is guaranteed to exist for $y \in B$, it is not guaranteed to be one-to-one unless $f$ is also injective. Therefore $f^{-1}(y)$ is a subset ($\subset$) of $A$, rather than an element in ($\in$) $A$.
\item \emph{False}. If $B$ is the codomain of $f$, and $D$ is a subset of $B$, then unless $f$ is surjective, there is no guarantee that $D$ intersects the image of $A$ in $B$.
\item \emph{True}. Let $f: \, A \rightarrow B$ and $g: \, B \rightarrow C$ be surjective functions. Since $g$ is surjective, the range of $g$ is equal to $C$, and for all $b \in B$, there exists $c \in C$. Since $f$ is surjective, the range of $f$ is equal to $B$, and for all $a \in A$, there exists $b \in B$. So $g \circ f(a)= g(f(a)) = g(b) =c$. Therefore $g \circ f$ is surjective.
\item \emph{True}. If $f$ is bijective, then its domain is $A$ and its range and codomain are $B$. Since $f$ is a function, each $x$ in $f$ corresponds to only one $y$ in $B$. Therefore, if $(x,y) \in f$, it must follow that the inverse of $f$ is bijective as well, and $(y,x) \in f^{-1}$.
\item \emph{False}. The identity function maps $\mathbb{R}$ onto $\mathbb{R}$ such that for all $x \in \mathbb{R}$, $(i)x = x$.
\end{enumerate}

\nsubsection{4}{Find all possible functions $f: \, A \rightarrow B$ in each case below. Describe the functions by listing their ordered pairs.}
\begin{enumerate}[(a)]
\item $ f = \set{(1,5), (2,5), (3,5)}$.
\item $ f_1 = \set{(4,5)}, f_2 = \set{(4,6)}$.
\item $ f_1 = \set{(1,5), (2,6)}, f_2 = \set{(1,5), (2,5)}, f_3=\set{(2,5),(1,6)}, f_4=\set{(2,5),(1,6)}$
\end{enumerate}

\nsubsection{6}{Let $A \subset \mathbb{R}$ and define $f: \, A \rightarrow B$ as given below. In each case describe an $A$ that is as large as possible while making $f$ injective.}
\begin{enumerate}[(a)]
\item $A = [5, \infty)$ or $A = (-\infty, 5]$.
\item $A = [-0.5, \infty)$ or $A = (-\infty, -0.5]$.
\item $A = [-\frac{\pi}{2}, \frac{\pi}{2}]$ or $A = [\frac{\pi}{2}, \frac{3\pi}{2}]$ (for example).
\end{enumerate}

\nsubsection{8}{Circles in the plane}
\begin{enumerate}[(a)]
\item Let $S$ be the set of all circles in the plane. Define $f: \, S \rightarrow [0, \infty)$ by $f(C)$ = the area of $C$ for all $C \in S$. \\
Assuming a circle can have a radius $r=0$, a function that maps all circles to the value of their area is surjective but not injective, because two circles at different coordinate positions can nonetheless have equal area.
\item Let $T$ be the set of all circles in the plane that are centered at the origin. Define $g: \, T \rightarrow [0, \infty)$ by $g(C) =$ the area of $C$ for all $C \in T$. \\
Now, $g$ is both injective and surjective; for every circle defined by radius length $r$ in the interval  $[0, \infty)$ there is an area $g(C)$ in $[0, \infty)$, and since all circles are centered at the origin, any circle with the same area has the same radius length and is the same circle.
\end{enumerate}

\nsubsection{10}{In each part, find a function $\mathbb{N}\rightarrow\mathbb{N}$ that has the desired properties.}
\begin{enumerate}[(a)]
\item Surjective, but not injective:
\begin{equation}
f(x) = 
\begin{cases}
x - 1, & \text{ for } x > 1, \\
1 & \text{ for } x = 1.

\end{cases}
\end{equation}

\item Injective, but not surjective:
\begin{equation}
f(x) = 2x.
\end{equation}
\item Neither surjective nor injective:
\begin{equation}
f(x) = 1
\end{equation}
\item Bijective:
\begin{equation}
f(x) = x.
\end{equation}
\end{enumerate}
\nsubsection{14}{Define $f: \, \mathbb{R} \rightarrow \mathbb{R}$ by $f(x) = x^2$, Find $f^{-1}(T)$ for each of the following.}
\begin{enumerate}[(a)]
\item $\set{-3, 3}$
\item $(-3, -2] \cup [2, 3)$
\item $[-3, 3]$
\end{enumerate}


\nsubsection{16}{Define $f: \, \mathbb{R}\rightarrow\mathbb{R}$ by $f(x) = x^2$. Find examples to show that equality does not hold in parts (a), (b), and (c) of Theorem 2.3.16.}
\begin{enumerate}[(a)]
\item Show that $C \neq f^{-1}[f(C)]$. \\
Let $C$  and $C'$ be nonempty subsets of $R$ such that $C \cap C' = \emptyset$ and $f(C) = f(C')$. In that case, $f^{-1}[f(C)] = C \cup C'$, and $C \subset C \cup C'$, so $C \neq f^{-1}[f(C)]$.

\end{enumerate}

\nsubsection{20}{Suppose that $f: \, A\rightarrow B$ and suppose that $C \subset A$ and $D \subset B$.}
\begin{enumerate}[(a)]
\item Prove or give a counterexample: $f(C) \subset D \iff C \subset f^{-1}(D)$. \\
Suppose $f(C) \subset D$ and suppose $x \in C$. Since $C \subset A$, $x \in A$. Since $x \in C$, $f(x) \in f(C)$ and therefore $f(x) \in D$, by supposition. \\
So $x \in A$, and $f(x) \in D$, meaning that $x \in f^{-1}(D)$, where $f^{-1}(D) = \{x \in C: f(x) \in D\}$. \\

From the other side, suppose $C \subset f^{-1}(D)$. Let $x \in C$. Then $x \in f^{-1}(D)$, and by the definition of inverse image, $f(x) \in D$.
\item What condition on $f$ will ensure that $f(C) = D \iff C = f^{-1}(D)$? \\
$f$ must be bijective.

\end{enumerate}
\section{Section 4}

%2.4 #2, 4, 6, 12, 16, 18, 20, 22 (8 problems)
\nsubsection{2}{Mark each statement True or False.}
\begin{enumerate}[(a)]
\item \emph{True}. This is the definition of ``equinumerous''.
\item \emph{False}. The empty set is also a finite set.
\item \emph{False}. A cardinal number that is not finite is transfinite.
\item \emph{False}. A set $S$ is denumerable if there exists a bijection $f: \, \mathbb{N}\rightarrow S$.
\item \emph{True}. See theorem 2.4.9.
\item \emph{False}. A set is denumerable iff it is equinumerous with $\mathbb{N}$. A subset of a denumerable set can be finite.
\end{enumerate}

\nsubsection{4}{Suppose that $m < n$. Prove that the intervals $(0,1)$ and $(m,n)$ are equinumerous by finding a specific bijection between them.}
Let $f(x)=(n-m)x + m$.

\nsubsection{6}{Prove: Every subset of a finite set is finite.}
Let $S$ be a finite set. If $S = \emptyset$, then its only subset is $\emptyset$, which is finite, and we are done. \\
Thus assume $S \neq \emptyset$, and there exists a bijection $f: \, I_n \rightarrow S$ for some $n \in \mathbb{N}$. If $T \subset S$ (and $T \neq \emptyset$) then let the elements of $I_n = \set{i_1, i_2, i_3, \ldots, n}$  in $T$ be represented by $i_1, i_2, i_3, \ldots, i_m$. \\
Define the function $g: \, T \rightarrow I_m$ by $(t, j) \in g$ where $(t, i_j) \in f$. Then $g$ is bijective, so $T$ is finite.

%\nsubsection{12}{Prove: Every infinite set is equinumerous with a proper subset of itself.}
%Suppose $S$ is an infinite set and suppose $T$ is a denumerable subset of $S$. 

%Define $f$ as an injection $f: \, T \rightarrow T$ such that $f(T) \neq T$.

%Therefore, $T$ is equinumerous with $S$.
%\nsubsection{16}{Determine whether each of the following is True or False.}
%\nsubsection{18}{S}
%\nsubsection{20}{}
%\nsubsection{22}{}
\end{document}