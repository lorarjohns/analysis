\documentclass[12pt]{scrartcl} % Font size

\input{structure.tex} 

\title{	
	\normalfont\normalsize
	\textsc{Analysis with an Introduction to Proof}\\ % Your university, school and/or department name(s)
	\vspace{25pt} % Whitespace
	\rule{\linewidth}{0.5pt}\\ % Thin top horizontal rule
	\vspace{20pt} % Whitespace
	{\huge Logic and Proof, Chapter Exercises}\\ % The assignment title
	\vspace{12pt} % Whitespace
	\rule{\linewidth}{2pt}\\ % Thick bottom horizontal rule
	\vspace{12pt} % Whitespace
}

\author{\LARGE L. J.} % Your name

\date{\normalsize\today} % Today's date (\today) or a custom date

\begin{document}

\maketitle % Print the title

\section{Section 1}

\subsection{Mark each statement true or false. Justify your answer.}

\begin{enumerate}[(a)]
	\item \emph{False.} A sentence need not be true in order to be classified a ``statement''. We require only that it ``can be clearly labeled'' true \emph{or} false. \\ We should specify the conditions that constitute clarity, and who's doing the labeling. We can condition truth and falsity on any number of assumptions---at best, explicit; at worst, implied---and the sentence's author ordinarily assumes a reader with a mental model that mirrors their own, for better or worse.
	\item \emph{False.} In binary logic, a statement cannot be true and false at the same time. Its truth value may be as-yet unknown---it might be a conjecture---but it has to follow the rules of mathematical proof. That is, to be a statement, the rules must be able to label a sentence exclusively ``true'' or ``false'' under \emph{some} set of specified conditions. Those conditions may change: $ x + 6 = 0 $ is a statement, for certain values of $x$, but not for all.
	\item \emph{True.} The negation of a statement need only have the opposite truth value to the statement itslf. So if $p$ is true, $\neg p$ is false, and $\neg(\neg p)$ is true.
	\item \emph{False.} The negation of a statement is by definition the logical complement of the given proposition. Thus, we can see that either $p$ can be false or $\neg p$ can be false, but they cannot both be false at the same time.
	\item \emph{True.} In logic, ``or'' is inclusive by default, and we must specify that we mean ``exclusive or'' ($\oplus$) (or its negation, the biconditional ($\iff$)) to denote $(p \lor q) \land \neg(p \land q)$.
\end{enumerate}	

\subsection{Mark each statement True or False. Justify each answer.}
\begin{enumerate}[(a)]
	\item \emph{False.} We refer to $p$ as the antecedent (that which comes (\emph{``ced-''}) before (\emph{``ante-''}). In contrast, $q$ is the consequent, or logical ``result''. In other words, $p$ is called the sufficient condition, and $q$ is called the necessary condition.
	\item \emph{True.} The implication applies only to worlds in which $p$ is true. Since we have only binary truth, we cannot rule out ``truth'' in worlds not containing $p$. Hence, these cases are true, and the only false one in that in which $p$ exists, but the necessary condition $q$ does not.\\
	\begin{center} 
	\begin{tabular}{*{2}{c}|*{1}{c}}$p$&$q$&$p \implies q$\\
		\hline
		T&T&T\\
		T&F&F\\
		F&T&T\\
		F&F&T\\
	\end{tabular}
	\end{center}
	\item \emph{False.} This formulation customarily evaluates to the same thing as ``$q$ whenever $p$'', because $p$ is a sufficient condition to ``trigger'' the requirement of $q$. In other words, whenever we have $p$, we must also have its necessary condition, $q$. But as the truth table in $(b)$ illustrates, the statement is true when $q$ is true but $p$ is false, and so the reverse dependency does not hold. Thus, it's incorrect to say that ``if $p$, then $q$'' is equivalent to ``$p$ whenever $q$''.
	\item \emph{True.} The negation of a conjunction is logically equivalent to the disjunction of the negations of its individual components.
	\begin{center}
	\begin{tabular}{*{2}{c}|*{3}{c}}$p$&$q$&$\neg(p \land q)$&$\iff$&$[(\neg p) \lor (\neg q)]$\\
	\hline
	T&T&F&T&F\\
	T&F&T&T&T\\
	F&T&T&T&T\\
	F&F&T&T&T\\
	\end{tabular}
	\end{center}
	
	\item \emph{False.} The negation of $p \implies q$ is not $q \implies p$.
	\begin{center}
	\begin{tabular}{*{2}{c}|*{3}{c}}$p$&$q$&$\neg(p \implies q)$&$
	\cancel{\iff} $&$q \implies p$\\
\hline
T&T&F&F&T\\
T&F&T&T&T\\
F&T&F&F&F\\
F&F&F&F&T\\
\end{tabular}\end{center}
\end{enumerate}
\subsection{Write the negation of each statement.}
\begin{enumerate}[(a)]
	\item The $3x3$ identity matrix is not singular.
	\item The function $f(x)=sin(x)$ is not bounded on $\mathbb{R}$.
	\item The functions $f$ and $g$ are nonlinear.
	\item Six is not prime and seven is even.
	\item $x$ is in $D$, and $f(x) \geq 5$.
	\item $(a_n)$ is monotone and bounded, and $(a_n)$ is divergent.
	\item $f$ is injective, but $S$ is not finite and $S$ is not denumerable.
\end{enumerate}
\subsection{Write the negation of each statement.}
\begin{enumerate}[(a)]
	\item $f(x)$ is not continuous at $x=3$.
	\item $R$ is not reflexive and not symmetric.
	\item $4$ and $9$ are not relatively prime.
	\item $x \notin A \lor x \in B$.
	\item $x < 7$ and $f(x)$ is in C.
	\item $(a_n)$ is convergent and $(a_n)$ is unbounded or not monotone.
	\item $f$ is continuous and $A$ is open, but $f^{-1}(A)$ is closed.
\end{enumerate}
\subsection{Identify the antecedent and the consequent in each statement.}
\begin{enumerate}[(a)]
	
	\item
	\begin{itemize}
		\item \emph{Antecedent:} $M$ is singular.
		\item \emph{Consequent:} $M$ has a zero eigenvalue.
	\end{itemize}
	
	\item
	\begin{itemize}
		\item \emph{Antecedent:} Linearity.
		\item \emph{Consequent:} Continuity.
	\end{itemize}
	
	\item
	\begin{itemize}
		\item \emph{Antecedent:} A sequence is Cauchy.
		\item \emph{Consequent:} It is bounded.
	\end{itemize}
	
	\item
	\begin{itemize}
		\item \emph{Antecedent:} $y > 5$.
		\item \emph{Consequent:} $x <3$.
	\end{itemize}
	\end{enumerate}

\subsection{Identify the antecedent and the consequent in each statement.}
\begin{enumerate}[(a)]
	\item
	\begin{itemize}
		\item \emph{Antecedent:} [A sequence] is Cauchy.
		\item \emph{Consequent:} [It] is convergent.
	\end{itemize}

	\item
	\begin{itemize}
		\item \emph{Antecedent:} Boundedness.
		\item \emph{Consequent:} Convergence.
	\end{itemize}
	
	\item 
	\begin{itemize}
		\item \emph{Antecedent:} Orthogonality.
		\item \emph{Consequent:} Invertibility.
	\end{itemize}
	
	\item 
	\begin{itemize}
		\item \emph{Antecedent:} $K$ is closed and bounded.
		\item \emph{Consequent:} $K$ is compact.
	\end{itemize}
\end{enumerate}

\subsection{Construct a truth table for each statement.}
\begin{enumerate}[(a)]
	\item 
		%\begin{center}
		\begin{tabular}{*{2}{c}|*{1}{c}}$p$&$q$&$p \implies \neg q$\\
		\hline
		T&T&F\\
		T&F&T\\
		F&T&T\\
		F&F&T\\
		\end{tabular}
		%\end{center}
	\item
		\begin{tabular}{*{2}{c}|*{3}{c}}$p$&$q$&$p \land (p \implies q)$&$\implies$&$q$\\
		\hline
		T&T&T&T&T\\
		T&F&F&T&F\\
		F&T&F&T&T\\
		F&F&F&T&F\\
		\end{tabular}
		
	\item
		\begin{tabular}{*{2}{c}|*{3}{c}}$p$&$q$&$p \implies (q \land \neg q)$&$\iff$&$\neg p$\\
		\hline
		T&T&F&T&F\\
		T&F&F&T&F\\
		F&T&T&T&T\\
		F&F&T&T&T\\
		\end{tabular}
\end{enumerate}

\subsection{Construct a truth table for each statement.}
\begin{enumerate}[(a)]
	\item
		\begin{tabular}{*{2}{c}|*{1}{c}}$p$&$q$&$p \lor \neg q$\\
		\hline
		T&T&T\\
		T&F&T\\
		F&T&F\\
		F&F&T\\
		\end{tabular}
	\item
		\begin{tabular}{*{1}{c}|*{1}{c}}$p$&$p \land \neg p$\\
		\hline
		T&F\\
		F&F\\
		\end{tabular}
	\item
		\begin{tabular}{*{2}{c}|*{3}{c}}$p$&$q$&$[(\neg q) \land (p \implies q)]$&$\implies$&$\neg p$\\
		\hline
		T&T&F&T&F\\
		T&F&F&T&F\\
		F&T&F&T&T\\
		F&F&T&T&T\\
		\end{tabular}
\end{enumerate}
\subsection{Indicate whether each statement is True or False.}
\begin{enumerate}[(a)]
	\item T $\land$ T $\implies$ T.
	\item F $\lor$ T $\implies$ T.
	\item F $\lor$ F $\implies$ F.
	\item (T $\implies$ T) $\implies$ T.
	\item (F $\implies$ F) $\implies$ T.
	\item (T $\implies$ F) $\implies$ F.
	\item {[}(T $\land$ F) $\implies$ T {]} $\implies$ T.
	\item {[}(T $\lor$ F) $\implies$ F {]} $\implies$ F.
	\item {[}(T $\land$ F) $\implies$ F {]} $\implies$ T.
	\item $\neg$(F $\lor$ T) $\iff$ T $\land$ F $\implies$ F.
	
\end{enumerate}
\subsection{Indicate whether each statement is True or False.}
\begin{enumerate}[(a)]
	\item T $\land$ F $\implies$ F.
	\item F $\lor$ F $\implies$ F.
	\item F $\lor$ T $\implies$ T.
	\item (T $\implies$ F) $\implies$ F.
	\item (F $\implies$ F) $\implies$ T.
	\item (F $\implies$ T) $\implies$ T.
	\item {[} (F $\lor$ T) $\implies$ F {]} $\implies$ F.
	\item {[} (T $\iff$ F) $\implies$ T {]} $\implies$ T.
	\item {[} (T $\land$ T) $\implies$ F {]} $\implies$ F.
	\item $\neg$(F $\land$ T) $\iff$ T $\lor$ F $\implies$ T.
\end{enumerate}

\subsection{Let $p$ be the statement ``The figure is a polygon,'' and let $q$ be the statement ``The figure is a circle''. Express each of the following statements in
symbols.}
\begin{enumerate}[(a)]
	\item $p \land \neg q$.
	\item $(p \lor q) \land \neg (p \land q)$.
	\item $\neg q \implies p$.
	\item $q \implies \neg p$.
	\item $p \iff \neg q$.

\end{enumerate}
\subsection{Let $m$ be the statement ``$x$ is perpendicular to $M$'', and let n be the statement
``$x$ is perpendicular to $N$''. Express each of the following statements in
symbols.}
\begin{enumerate}[(a)]
	\item $n \land \neg m$.
	\item $\neg (m \lor n)$.
	\item $n \implies m$
	\item $m \implies \neg n$
	\item $ \neg (m \land n) \iff \neg m \lor \neg n$
\end{enumerate}

\subsection{Define ``nor'' ($\nabla$) as:}

\begin{center}
\begin{tabular}{*{2}{c}|*{1}{c}}$p$&$q$&$p \nabla q$\\
\hline
T&T&F\\
T&F&F\\
F&T&F\\
F&F&T\\
\end{tabular}
\end{center}

\begin{enumerate}[(a)]
	\item \begin{tabular}{*{1}{c}|*{3}{c}}$p$&$\neg p$&$\iff$&$p \nabla p$\\
			\hline	
			T&F&T&F\\
			F&T&T&T\\
			\end{tabular}	
	\item \begin{tabular}{*{2}{c}|*{3}{c}}$p$&$q$&$(p \nabla p)$&$\iff$&$(q \nabla q)$\\
			\hline
			T&T&F&T&F\\
			T&F&F&F&T\\
			F&T&T&F&F\\
			F&F&T&F&T\\
		  \end{tabular}
		  
	\item \begin{tabular}{*{2}{c}|*{1}{c}}$p$&$q$&$p \land q$\\
			\hline
			T&T&T\\
			T&F&F\\
			F&T&F\\
			F&F&F\\
		  \end{tabular}
\end{enumerate}

\subsection{Use truth tables to verify that each of the following is a tautology.}

\emph{Commutative laws}

\begin{enumerate}[(a)]
	\item \begin{tabular}{*{2}{c}|*{3}{c}}$p$&$q$&$(p \land q)$&$\iff$&$(q \land p$\\
\hline
T&T&T&T&T\\
T&F&F&T&F\\
F&T&F&T&F\\
F&F&F&T&F\\	
\end{tabular}

	\item \begin{tabular}{*{2}{c}|*{3}{c}}$p$&$q$&$(p \lor q)$&$\iff$&$(q \lor p$\\
\hline
T&T&T&T&T\\
T&F&T&T&T\\
F&T&T&T&T\\
F&F&F&T&F\\
\end{tabular}

\emph{Associative laws}
	\item \begin{tabular}{*{3}{c}|*{3}{c}}$p$&$q$&$r$&$[p \land (q \land r)]$&$\iff$&$[(p \land q) \land r$\\
\hline
T&T&T&T&T&T\\
T&T&F&F&T&F\\
T&F&T&F&T&F\\
T&F&F&F&T&F\\
F&T&T&F&T&F\\
F&T&F&F&T&F\\
F&F&T&F&T&F\\
F&F&F&F&T&F\\
\end{tabular}

	\item \begin{tabular}{*{3}{c}|*{3}{c}}$p$&$q$&$r$&$[p \lor (q \lor r)]$&$\iff$&$[(p \lor q) \lor r$\\
\hline
T&T&T&T&T&T\\
T&T&F&T&T&T\\
T&F&T&T&T&T\\
T&F&F&T&T&T\\
F&T&T&T&T&T\\
F&T&F&T&T&T\\
F&F&T&T&T&T\\
F&F&F&F&T&F\\
\end{tabular}

\emph{Distributive laws}
\item \begin{tabular}{*{3}{c}|*{3}{c}}$p$&$q$&$r$&$[p \land (q \lor r)]$&$\iff$&$[(p \land q) \lor (p \land r)$\\
\hline
T&T&T&T&T&T\\
T&T&F&T&T&T\\
T&F&T&T&T&T\\
T&F&F&F&T&F\\
F&T&T&F&T&F\\
F&T&F&F&T&F\\
F&F&T&F&T&F\\
F&F&F&F&T&F\\
\end{tabular}

\item \begin{tabular}{*{3}{c}|*{3}{c}}$p$&$q$&$r$&$[p \lor (q \land r)]$&$\iff$&$[(p \lor q) \land (p \lor r)$\\
\hline
T&T&T&T&T&T\\
T&T&F&T&T&T\\
T&F&T&T&T&T\\
T&F&F&T&T&T\\
F&T&T&T&T&T\\
F&T&F&F&T&F\\
F&F&T&F&T&F\\
F&F&F&F&T&F\\
\end{tabular}
\end{enumerate}

\section{Section 2}

\nsubsection{2}{Mark each statement true or false. Justify your answer.}
\begin{enumerate}[(a)]
	\item \emph{False}. $\exists$ denotes the existence at least one. $\forall$ is the universal quantifier and means ``for each, for every''.
	\item \emph{True}. Universal quantification is assumed if a variable is used in the antecedent of an implication without an explicit quantifier.
	\item \emph{True}. For instance, $\forall x \exists y$ corresponds to the statement ``For every disease, there exists a cure'', while $\exists x \forall y$ corresponds to the statement ``There is a panacea for all diseases.''	
\end{enumerate}

\nsubsection{4}{Write the negation of each statement.}
\begin{enumerate}[(a)]
	\item Someone does not like Robert.
	\item No students work part-time.
	\item Some square matrix is triangular.
	\item $\forall x \in B, f(x) \leq k$.
	\item $\exists x \ni (x > 5) \land [(f(x) \geq 3) \land (f(x) \leq 7)]$.
	\item $\exists x \in A \ni \forall y \in B, f(x) \geq f(y)$.
\end{enumerate}

\nsubsection{6}{Determine the truth value of each statement, assuming $x$ is a real number.
Justify your answer.}
\begin{enumerate}[(a)]
	\item \emph{True}. Let $x = 4$. $3 < 4$ and $4 < 5$, so $4$ is included in the interval.
	\item \emph{False}. Let $x = 3$ Then $3 \in [3, 5]$ and $3 < 4$.
	\item \emph{True}. Let $x = 3$. Then $3^2 = 9 \neq 3$.
	\item \emph{False}. Let $x = \sqrt{3}$. Then $x$ is a real number and $(\sqrt{3})^2 = 3$.
	\item \emph{False}.
	Let $x = 0$. Then $0^2 = 0$. \\
	Let $x > 0$. Then for all $x$, $x^2$ is positive. \\
	Let $x < 0$. Then for all $x$, $x^2$ is also positive. \\
	Therefore, for all real $x$, $x \neq -5$.
	\item \emph{False}. Let $ x = -5$. Then $(-5)^2 = 25 \neq -5$.
	\item \emph{True}. Let $x = x$. $x - x = 0 \iff x = x$. (This is the additive identity of a field.)
	\item \emph{True}. By the same token, it is always true that for all $x$, $x - x = 0$.
\end{enumerate}

\nsubsection{8}{Which of the following best identifies f as a constant function, where x and y are real numbers?}
\begin{itemize}
	\item[] $\exists y \ni \forall x, f(x) = y$ represents a constant function.
\end{itemize}

\nsubsection{10}{Determine the truth value of each statement, assuming that x and y are realnumbers. Justify your answer.}
\begin{enumerate}[(a)]
	\item \emph{True}. For all $x \in \mathbb{R}$, let $y = 0$. Then $xy = 0$.
	\item \emph{False}. Let $x = 0$. Then for all $y$, $xy = 0 \neq 1$.
	\item \emph{False}. For all $y \in \mathbb{R}$, there is a real number $x = 0$ such that $xy = 0(y) = 0 \neq 1$.
	\item \emph{True}. Let $y = 1$. Then for all $x \in \mathbb{R}$, $xy = x(1) = x$.
\end{enumerate}

\nsubsection{12}{Determine the truth value of each statement, assuming that x, y, and z are real numbers. Justify your answer.}
\begin{enumerate}[(a)]
	\item \emph{True}. Suppose by contradiction that for all $x, y, z \in \mathbb{Z}$, there exist some $x$ and $y$ such that $x + y \neq z$. \\Then some integer would have no successor, which violates the properties of $\mathbb Z$. \\So there must be some $x$ and $y$ such that $x + y = z$.
	\item \emph{False}. Let $x = 0$.\\ Then, for any $y$, let $z = y + 0$, or $z = y$.\\
	Let $z = 1$. Since $y \neq 1$ for all $y \in \mathbb{R}$, the statement is false.
	\item \emph{True}. Let $x$ be any real number other than zero. Then, given any $y$, let $z = \frac{y}{x}$.
	\item \emph{False}. Let $x = 2$. Then $z \leq 2 + y$.\\
	Then for all $y$, we must show there exists a $z$ such that $y < z \leq x + y$, or $0 < z - y \leq x$.\\
	Assuming $x = 2$, we must show there is some $z$ such that $y < z \leq 2 + y$. \\Let $z = y + 1$. Then $y < y + 1 < y + 2$. Therefore, the statement is false, because $z < x + y$.
	\item \emph{False}. Assume by contradiction that $y < z \leq x + y$, or $0 < z - y \leq x$.\\
	Let $x = 2$. Then for all $z$, $y < z \leq y + 2$.\\
	Let $y = z - 1$, such that $z - 1 < z \leq z + 1$.\\
	Then $z > z - 1$ and $z \leq 2 + (z - 1) = z + 1$ for all $z$.\\
	Since the negation is true, the statement is false.
\end{enumerate}
\nsubsection{14}{Rewrite the defining conditions in logical symbolism, and write the negation.}
\begin{enumerate}[(a)]
	\item $\exists k > 0 \ni \forall x, f(x + k) = f(x) \implies \text{f is periodic}$.
	\item $\neg(\text{f is periodic}) \implies \forall x > 0, \exists x \ni f(x + k) \neq f(x)$.
\end{enumerate}
\nsubsection{16}{Rewrite the defining conditions in logical symbolism, and write the negation.}
\begin{enumerate}[(a)]
	\item $[\forall x, y, x < y \implies f(x) > f(y)] \implies f \text{is strictly decreasing}$.
	\item $\neg(f \text{is strictly decreasing}) \implies \exists x, y \ni (x < y) \land f(x) \leq f(y)$.
\end{enumerate}
\nsubsection{18}{Rewrite the defining conditions in logical symbolism, and write the negation.}
\begin{enumerate}[(a)]
	\item $[\forall y \in B \exists x \in A \ni f(x) = y] \implies f \text{is surjective}.$
	\item $\neg(f \text{is surjective}) \implies \exists y \in B \ni \forall x \in A, f(x) \neq y$.
\end{enumerate}
\nsubsection{20}{Rewrite the defining conditions in logical symbolism, and write the negation.}
\begin{enumerate}[(a)]
	\item $\forall \epsilon > 0 \exists \delta > 0 \ni \forall x, y \in S, \lvert x - y \rvert < \delta \implies \lvert f(x) - f(y) \rvert < \epsilon$.
	\item $\exists \epsilon > 0 \ni \forall \delta > 0, \exists x, y \in S \ni (\lvert x - y \rvert < \delta) \land (\lvert f(x) - f(y) \rvert \geq \epsilon.$
\end{enumerate}

\end{document}