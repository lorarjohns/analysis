\documentclass[12pt]{scrartcl} % Font size

%%%%%%%%%%%%%%%%%%%%%%%%%%%%%%%%%%%%%%%%%
% Original from:
% Frits Wenneker
% CC BY-NC-SA 3.0 (http://creativecommons.org/licenses/by-nc-sa/3.0/)
%%%%%%%%%%%%%%%%%%%%%%%%%%%%%%%%%%%%%%%%%

\usepackage{amsmath, amsfonts, amsthm} % Math packages

\usepackage{cancel}

\usepackage[shortlabels]{enumitem}

\usepackage{listings} % Code listings, with syntax highlighting

\usepackage[english]{babel} % English language hyphenation

\usepackage{graphicx} % Required for inserting images
\graphicspath{{Figures/}{./}} % Specifies where to look for included images (trailing slash required)

\usepackage{booktabs} % Required for better horizontal rules in tables

\numberwithin{equation}{section} % Number equations within sections (i.e. 1.1, 1.2, 2.1, 2.2 instead of 1, 2, 3, 4)
\numberwithin{figure}{section} % Number figures within sections (i.e. 1.1, 1.2, 2.1, 2.2 instead of 1, 2, 3, 4)
\numberwithin{table}{section} % Number tables within sections (i.e. 1.1, 1.2, 2.1, 2.2 instead of 1, 2, 3, 4)

\setlength\parindent{0pt} % Removes all indentation from paragraphs

\setlist{noitemsep} % No spacing between list items


%	DOCUMENT MARGINS

\usepackage{geometry} % Required for adjusting page dimensions and margins

\geometry{
	paper=a4paper, % Paper size, change to letterpaper for US letter size
	top=2.5cm, % Top margin
	bottom=3cm, % Bottom margin
	left=3cm, % Left margin
	right=3cm, % Right margin
	headheight=0.75cm, % Header height
	footskip=1.5cm, % Space from the bottom margin to the baseline of the footer
	headsep=0.75cm, % Space from the top margin to the baseline of the header
	%showframe, % Uncomment to show how the type block is set on the page
}

%	FONTS

\usepackage[utf8]{inputenc} % Required for inputting international characters
\usepackage[T1]{fontenc} % Use 8-bit encoding

\usepackage{fourier} % Use the Adobe Utopia font for the document

%	SECTION TITLES

\usepackage{sectsty} % Allows customising section commands

\sectionfont{\vspace{6pt}\centering\normalfont\scshape} % \section{} styling
\subsectionfont{\normalfont\bfseries} % \subsection{} styling
\subsubsectionfont{\normalfont\itshape} % \subsubsection{} styling
\paragraphfont{\normalfont\scshape} % \paragraph{} styling

%	HEADERS AND FOOTERS

\usepackage{scrlayer-scrpage} % Required for customising headers and footers

\ohead*{} % Right header
\ihead*{} % Left header
\chead*{} % Centre header

\ofoot*{} % Right footer
\ifoot*{} % Left footer
\cfoot*{\pagemark} % Centre footer

% COMMANDS AND SYMBOLS

\newcommand{\enumalpha}{
	\begin{enumerate}[label=(\alph*)]
		
	} 

\title{	
	\normalfont\normalsize
	\textsc{Analysis with an Introduction to Proof}\\ % Your university, school and/or department name(s)
	\vspace{25pt} % Whitespace
	\rule{\linewidth}{0.5pt}\\ % Thin top horizontal rule
	\vspace{20pt} % Whitespace
	{\huge Logic and Proof, Chapter Exercises}\\ % The assignment title
	\vspace{12pt} % Whitespace
	\rule{\linewidth}{2pt}\\ % Thick bottom horizontal rule
	\vspace{12pt} % Whitespace
}

\author{\LARGE L. J.} % Your name

\date{\normalsize\today} % Today's date (\today) or a custom date

\begin{document}

\maketitle % Print the title

\section{Section 1}

\subsection{Mark each statement true or false. Justify your answer.}

\begin{enumerate}[(a)]
	\item \emph{False.} A sentence need not be true in order to be classified a ``statement''. We require only that it ``can be clearly labeled'' true \emph{or} false. \\ We should specify the conditions that constitute clarity, and who's doing the labeling. We can condition truth and falsity on any number of assumptions---at best, explicit; at worst, implied---and the sentence's author ordinarily assumes a reader with a mental model that mirrors their own, for better or worse.
	\item \emph{False.} In binary logic, a statement cannot be true and false at the same time. Its truth value may be as-yet unknown---it might be a conjecture---but it has to follow the rules of mathematical proof. That is, to be a statement, the rules must be able to label a sentence exclusively ``true'' or ``false'' under \emph{some} set of specified conditions. Those conditions may change: $ x + 6 = 0 $ is a statement, for certain values of $x$, but not for all.
	\item \emph{True.} The negation of a statement need only have the opposite truth value to the statement itslf. So if $p$ is true, $\neg p$ is false, and $\neg(\neg p)$ is true.
	\item \emph{False.} The negation of a statement is by definition the logical complement of the given proposition. Thus, we can see that either $p$ can be false or $\neg p$ can be false, but they cannot both be false at the same time.
	\item \emph{True.} In logic, ``or'' is inclusive by default, and we must specify that we mean ``exclusive or'' ($\oplus$) (or its negation, the biconditional ($\iff$)) to denote $(p \lor q) \land \neg(p \land q)$.
\end{enumerate}	

\subsection{Mark each statement True or False. Justify each answer.}
\begin{enumerate}[(a)]
	\item \emph{False.} We refer to $p$ as the antecedent (that which comes (\emph{``ced-''}) before (\emph{``ante-''}). In contrast, $q$ is the consequent, or logical ``result''. In other words, $p$ is called the sufficient condition, and $q$ is called the necessary condition.
	\item \emph{True.} The implication applies only to worlds in which $p$ is true. Since we have only binary truth, we cannot rule out ``truth'' in worlds not containing $p$. Hence, these cases are true, and the only false one in that in which $p$ exists, but the necessary condition $q$ does not.\\
	\begin{center} 
	\begin{tabular}{*{2}{c}|*{1}{c}}$p$&$q$&$p \implies q$\\
		\hline
		T&T&T\\
		T&F&F\\
		F&T&T\\
		F&F&T\\
	\end{tabular}
	\end{center}
	\item \emph{False.} This formulation customarily evaluates to the same thing as ``$q$ whenever $p$'', because $p$ is a sufficient condition to ``trigger'' the requirement of $q$. In other words, whenever we have $p$, we must also have its necessary condition, $q$. But as the truth table in $(b)$ illustrates, the statement is true when $q$ is true but $p$ is false, and so the reverse dependency does not hold. Thus, it's incorrect to say that ``if $p$, then $q$'' is equivalent to ``$p$ whenever $q$''.
	\item \emph{True.} The negation of a conjunction is logically equivalent to the disjunction of the negations of its individual components.
	\begin{center}
	\begin{tabular}{*{2}{c}|*{3}{c}}$p$&$q$&$\neg(p \land q)$&$\iff$&$[(\neg p) \lor (\neg q)]$\\
	\hline
	T&T&F&T&F\\
	T&F&T&T&T\\
	F&T&T&T&T\\
	F&F&T&T&T\\
	\end{tabular}
	\end{center}
	
	\item \emph{False.} The negation of $p \implies q$ is not $q \implies p$.
	\begin{center}
	\begin{tabular}{*{2}{c}|*{3}{c}}$p$&$q$&$\neg(p \implies q)$&$
	\cancel{\iff} $&$q \implies p$\\
\hline
T&T&F&F&T\\
T&F&T&T&T\\
F&T&F&F&F\\
F&F&F&F&T\\
\end{tabular}\end{center}
\end{enumerate}
\subsection{Write the negation of each statement.}
\begin{enumerate}[(a)]
	\item The $3x3$ identity matrix is not singular.
	\item The function $f(x)=sin(x)$ is unbounded on $\mathbb{R}$.
	\item The functions $f$ and $g$ are nonlinear.
	\item Six is not prime and seven is even.
	\item $x$ is in $D$, and $f(x) \geq 5$.
	\item $(a_n)$ is monotone and bounded, and $(a_n)$ is divergent.
	\item $f$ is injective, but $S$ is not finite and $S$ is not denumerable.
\end{enumerate}
\subsection{Write the negation of each statement.}
\begin{enumerate}[(a)]
	\item $f(x)$ is discontinuous at $x=3$.
	\item $R$ is not reflexive and not symmetric.
	\item $4$ and $9$ are not relatively prime.
	\item $x \notin A \lor x \in B$.
	\item $x < 7$ and $f(x)$ is in C.
	\item $(a_n)$ is convergent and $(a_n)$ is unbounded or not monotone.
	\item $f$ is continuous and $A$ is open, but $f^{-1}(A)$ is closed.
\end{enumerate}
\subsection{Identify the antecedent and the consequent in each statement.}
\begin{enumerate}[(a)]
	
	\item
	\begin{itemize}
		\item \emph{Antecedent:} $M$ is singular.
		\item \emph{Consequent:} $M$ has a zero eigenvalue.
	\end{itemize}
	
	\item
	\begin{itemize}
		\item \emph{Antecedent:} Linearity.
		\item \emph{Consequent:} Continuity.
	\end{itemize}
	
	\item
	\begin{itemize}
		\item \emph{Antecedent:} A sequence is Cauchy.
		\item \emph{Consequent:} It is bounded.
	\end{itemize}
	
	\item
	\begin{itemize}
		\item \emph{Antecedent:} $y > 5$.
		\item \emph{Consequent:} $x <3$.
	\end{itemize}
	\end{enumerate}

\subsection{Construct a truth table for each statement.}
\begin{enumerate}[(a)]
	\item 
		%\begin{center}
		\begin{tabular}{*{2}{c}|*{1}{c}}$p$&$q$&$p \implies \neg q$\\
		\hline
		T&T&F\\
		T&F&T\\
		F&T&T\\
		F&F&T\\
		\end{tabular}
		%\end{center}
	\item
		
\end{enumerate}

\subsection{Identify the antecedent and the consequent in each statement.}

\section{Section 2}

\subsection{*}

\end{document}